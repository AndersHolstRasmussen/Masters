\chapter{Conclusion}
With coincidence measurements of \be-delayed \al-\al breakup of \li, we have successfully measured the excitation spectrum of \ber. This spectrum has been compared to previous more precise measurements, and a nice consensus between the shape of the two measurements are shown. 
\\
\\
The primary goal of the experiment was to measure the \be-\al\ angular correlations in the decay of \li, since it is previously shown that this would be isotropic. 
Two decays was measured using this setup, \li and \isotope[12]{B}, which involves a triple \al-decay, and a \be-decay. The setup was designed to measure \al-particles position and energy with high resolution, as well as detecting \be-particles position to equal resolution. 
\\
By comparing the excitation spectra of \ber, to previous measurements, we have shown that the setup has a satisfied energy resolution for \al-particles. This has been further enhanced as it is shown that a good coincidence sorting will give a more precise \al\ energy resolution.\\
When it comes to the \be-\al\ angular correlation, we have shown that there are some inconsistencies in the setup, that causes the angular analysis to more complex than first assumed. There is a good indication of the beam not hitting the exact center of the target, but there is also evidence of some detectors not being placed in a perfect cube around the target. \\
The target holder also casts a shadow on two of the detectors, which can further skew the measured angles away from the expected.