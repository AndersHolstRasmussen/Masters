\chapter{Conclusion and outlook}
The primary goal of the experiment was to measure the \be-\al\ angular correlations in the decay of \li, since it is previously shown that this would be isotropic. 
The experiment consisted of two parts, one measuring the decay of \li, and the main experiment measuring the decay of \isotope[12]{B}, which involves a triple \al-decay, and a \be-decay. The setup was designed to measure \al-particles' position and energy with high resolution, as well as detecting \be-particles' position to equal resolution. 
\\
By comparing the excitation spectra of \ber, to previous measurements, we have shown that the setup has a satisfactory energy resolution for \al-particles. This has been further enhanced as it is shown that a good sorting of particles in coincidence will give a more precise \al-energy resolution.\\
When it comes to the \be-\al\ angular correlation, we have shown that there are some inconsistencies in the setup, that causes the angular analysis to be more complex than first assumed. There is a good indication of the beam not hitting the exact center of the target, but there is also evidence of some detectors not being placed in a perfect cube around the target. \\
The target holder also casts a shadow on two of the detectors, which can further skew the measured angles away from the expected.\\
In regards to the cuts, the most significant cut was the one on maximum total momentum of both \al-particles. It was chosen on the basis that it seems to be a good cutoff for the peak of total momentum. But there seems to still be mislabeled particles after the sorting. 
By doing a simulation of the experiment, one can get a better insight in what a good cutoff should be. \\
Another improvement would also be to make a thorough analysis of what detectors had dead strips. There where quite a few, and they can influence the expected geometry.
Then there is also the position of the detectors, which doesn't seem to be placed exactly where we expected them to be. There could be some optimization in finding the exact positions, and then make an angular efficiency again. 
\\
Since we had to make the decision to only measure \be-particles in two out of 6 directions, an improvement would of course be to ensure that it would be possible to detect the \be-particles in every direction, and get a more complete picture of the angular correlations. This might be done with a better data sorting, accompanied by simulations. 