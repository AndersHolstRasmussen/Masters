\chapter{Analysis}
\section{The effects of the cuts}
How does the cuts imposed in the previous chapter affect the data then?
First we need to look at how the data looks, without any cuts. On \cref{fig:EENoCuts} there are the energy of the first \al-particle (E$_{\alpha1}$) is plotted agains the energy of the second \al-particle (E$_{\alpha2}$).
This gives us a nice view of what is considered \al-particle pairs. There is a prominent line going diagonally through the graph, where both particles have around the same energy. This line is expected, as the \al-particles will have close to equal energy, when decaying from \ber. 

This means that there is a lot of particle pairs that have been identified as \al-\al, but properbly was \al-noise, noise-noise, \al-\be\ or \be-\be\ pairs. 

The lines occuring from E$_{\alpha2} \approx \SI{400}{keV}$ are a clear example of a \al-\be\ pair. In this line, \al1 has been identified correctly as a \al-particle, which will have energies ranging from 500-\SI{6000}{keV}, whereas the the other particle has a smaller range, 

\begin{figure}[h]
	\includegraphics[width=\linewidth]{../figures/EENoCuts.pdf}
	\caption{The energy of the first \al-particle (E$_{\alpha1}$) against the energy of the second \al-particle (E$_{\alpha2}$). The intensity scale is in logarithmic, to get a better view of all the different particle configurations.}
	\label{fig:EENoCuts}
\end{figure}

\section{The properties of the \al-particles}


\section{Angular efficiency of the setup}
Since the detectors are unable to cover the entire solid angle, there will always be some mutual angles that are more likely to be measured. 
If we only look at one detector, a very large number of angles are not covered, but small mutual angles such as $\theta \approx 0$, are very easy to measure, as it is just a measurement of two particles in the same pixel. 
This effect becomes apparent on \cref{fig:oneDetEff}, where the angular efficiency is shown for Det2. For a single detector, there cannot be angles of $\theta = 90$ and higher, as the detector is flat. In this detector, the lowest angle found was $\cos(0.21) = 55\degree$.\\

\begin{figure}[h]
	\includegraphics[width=\linewidth]{../figures/det2Eff.pdf}
	\caption{The angular efficiency of a single detector (Det2). For one flat detector, there can never be angles of $90\degree$ and higher.}
	\label{fig:oneDetEff}
\end{figure}
In this setup however, we a cube of square detectors, who's normal vectors are all pointing in towards target at the center. This gives a much larger coverage of all mutual angles. 
The placement of the detectors gives that angles around $\theta \approx 90 \degree$ are also very favored. This makes sense, almost no matter what pixel was hit, there is a corresponding pixel $90\degree$ to both sides. In the same way, will there always be a corresponding pixel $\approx 180\degree$ from each pixels. This effect can be seen on \cref{fig:effAllDet}. \\
This histogram was created by using the spacial coordinates of the entire setup. First the positions of each pixel in each detector was found. Then two loops running over each pixel pair $i, j$, finds the angle between these pixels and saves it.\\

\begin{figure}[h]
	\includegraphics[width=\linewidth]{../figures/allDetEff.pdf}
	\caption{Two normalized histograms of the angular efficiency of the entire setup. The red histogram does not account for the effective area of a pixel. The blue is the true angular efficiency.}
	\label{fig:effAllDet}
\end{figure}
There is still a geometric effect that is not accounted for in the above analysis. We still need to consider that not all pixels in the detector has the same effective area. 
A pixel furthest out in a detector will have a effective area smaller than the area of a pixel in the center. This effect can be seen on \cref{fig:MexiHatDetector}. Here we see that there is a much higher count of particles hitting the center of the detector, and fewer hitting the edges. A white line crossing through the middle is a defect strip, which did not measure anything. Sadly, some of the detectors had defect strips, but on a large scale it was not very noticeable.
\begin{figure}[h]
	\begin{subfigure}[t]{.5\linewidth}
		\centering
		\includegraphics[width=\linewidth]{../figures/mexihatDet2.pdf}
		\caption{A plot over the number of hits in each strip for Det2. The white line in the middle is a defect strip, that did not measure anything. }
		\label{fig:MexiHatDetector}
	\end{subfigure}
	\begin{subfigure}[t]{.5\linewidth}
		\centering
		\includegraphics[width=\linewidth]{../figures/mexihatDet2THEORY.pdf}
		\caption{A theoretical intensity Det2. }
		\label{fig:MexiHatDetectorTheory}
	\end{subfigure}
\end{figure}

To account for this effect, each pixel will be associated with a corresponding area-efficiency $(\text{Eff}_A)$.
This is calculated as 
\begin{equation}
\text{Eff}_A = \dfrac{\cos(\theta)}{r^2 },
\end{equation}
where $r$ is the distance to the pixel and $\theta$ is the angle between the inverted normal vector of the pixel and the line from the center to the pixel. A illustration of the scenario can be seen on \cref{fig:EffGeometry}.

On \cref{fig:effwithweight} two histograms can be seen. The red line represents the angular efficiency of the setup, without accounting for the relative area of the pixels, while the blue line is a weighted histogram for the same angles, with each pixels relative area accounted for. \\
The form of the two histograms are quite similar around 1, 0 and $-1$ but in between there is a rather prominent difference. Therefore it is important that the effective area of the pixel is accounted for, when we in \cref{sec:betaAngle} will look at the angular correlations of the \be-particle in the setup. 



\section{Angular correlations of \al-particles and \be-particles}
\label{sec:betaAngle}
From what we know in \note{ref til beta = isotrop} the \be-particles must have an isotropic distribution. Since we only have two detectors in the setup that are capable of measuring \be-particles, we cannot measure the angle from one particle to something constant, i.e. the beam. Therefore we measure the angle between both \al-particles and the \be-particles.
This is done by first finding the angle between the first \al-particle \al$_1$, and create a histogram of this. Then the angle between the second \al-particle \al$_2$ is measured, and a histogram is created. The two histograms are then added, to get the full picture of the mutual angles of the particles. 
This can be seen on \cref{fig:effwithweight} as the green line.  

The blue line is this figure is the angular efficiency for the specific case, where the \al-particles can hit in any given detector, but the \be-particles can only hit in the two specific detectors Det2 and DetD??. The two histograms has both been normalized, for a better comparison. By dividing the two histograms with each other, we get a better comparison. If they where to be close to equal, we would see a flat curve. But what we see on \cref{fig:dataDivEff} is not very flat. 
There are small fluctuations in the line, which stems from the inaccuracy of calculated angular efficiency. But the main shape of the curve is at most time not around $y=1$.

\begin{figure}[h]
	\centering
	\includegraphics[width=\linewidth]{../figures/betaAngles/betaAngle.pdf}
	\caption{Some figure of the efficiency of the setup without efficiency of each pixel. }
	\label{fig:effwithweight}
\end{figure}

\begin{figure}[h]
	\centering
	\includegraphics[width=\linewidth]{../figures/betaAngles/dataDivEff.pdf}
	\caption{Data divided by the angular efficiency. }
	\label{fig:dataDivEff}
\end{figure}

There are two different explanations as to why this might be.
The first explanation is that the beam did not hit the target in the center. The calculated efficiency assumes that the beam hits precisely in the middle, but in any setup, there can be a few millimeters of errors. 

By finding the angular efficiency of the setup, with different values for the center and comparing this to the measured angles, we have found a slight correction for the center of the beam. This can be seen on \cref{fig:centerCorrection}, where the beam has been moved to the coordinates $(-3, -3, 0)$, as opposed to the previous of $(0, 0, 0)$. The division fo the two histograms on \cref{fig:dataDivEffCenter} shows a more stable line from $\cos(-0.4)$ to $\cos(0.4)$, but the edges are still not very alike. Looking at $\cos(1)$, we see a rapid fall, which indicates that we have measured a lot fewer parallel \be-particles and \al-particles than the setup is designed to handle. 
This can be due to the fact that some of the strips where defect. The calculated angular efficiency does not know which strips where defect, and and will therefore have a lot higher efficiency for parallel particles. The defect strips will therefore also play a role in the grand scheme, and is a possible explanation as to why there is a difference in the calculated and the measured data. 
\begin{figure}[h]
	\begin{subfigure}[]{.5\linewidth}
		\centering
		\includegraphics[width=\linewidth]{../figures/centerCorrections/try-3-30.png}
		\caption{The angular efficiency and the data, with a correction for the center of the beam. }
		\label{fig:centerCorrection}
	\end{subfigure}
	\begin{subfigure}[]{.5\linewidth}
		\centering
		\includegraphics[width=\linewidth]{../figures/betaAngles/dataDivEffCenterCorrected.pdf}
		\caption{Measured \be- \al-particle angular distribution divided by the calculated angular efficiency of the setup, with a correction for position of the beam. }
		\label{fig:dataDivEffCenter}
	\end{subfigure}
\end{figure}

%A way to check if the coordinates $(-3, -3, 0)$ is a good fit, we can compare the distribution of particle hits, with the calculated distribution found with this new center. This has been done on \cref{fig:mexihatCompare}.


