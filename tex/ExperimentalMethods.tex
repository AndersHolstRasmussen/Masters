\chapter{Experimental Methods}
The data used in this thesis stems from an experiment carried out at the IGISOL facility, at the University of Jyväskylä. Originally the experiment was scheduled for May 2020, but was postponed due to the ongoing Corona pandemic. 


\section{Experimental setup}
The setup is designed to measure $\beta\alpha$ angular correlations in the $\beta$-delayed particle decay of \isotope[8][]{Li}. When measuring multiple particles, the setup is highly dependent on the coverage of the solid angle. Therefore the setup is designed to have a large solid angle coverage, with high $\alpha$-particle resolution, while still being able to measure $\beta$-particles. \\
This is has been achieved by creating a cube of six double sided silicon detectors (DSSD), all backed by a \SI{1}{mm} single sided detector (SSD). To gain the largest solid angle, the detectors where placed as close to one another as possible. A 3D printed case was designed to hold the detectors in place, and achieved a solid angle coverage of 51\% for the DSSD's. An illustration of the setup, together with the different detectors' thickness can be seen on figure \ref{fig:opstilling}. 
Even though the setup was designed to hold 12 detectors in total, there where only 11 detectors in the actual experiment. This was due to one of the SSD's being defect, so it was removed from the setup. 

\begin{figure}[h]
	\includegraphics[width=\linewidth]{../figures/cosang.pdf}
	\caption{illustration of the setup. Note that the SSD P1 was absent from the actual experiment, due to failure in the detector}
	\label{fig:opstilling}
\end{figure}

\section{The detectors}
As mentioned above, there where two types of detectors present in the setup. The first type is the Double sided silicon detector. 
As the name suggests, it consists of two sides, a front layer and a back layer. Each layer consists of 16 strips, that are placed in rows next to each other. The two layers are then arranged so each side are mutually orthogonal, which defectively makes pixels where each strip intersects a strip on the other side. \\
The strips on the front side are p-doped, while the back side are n-doped. When a charged particle hits the detector, it will ionize the atoms in the semi-conductor, and produce a electron-hole pair. The number of electron-hole pairs is proportional to the energy of the charged particle. 
The bias voltage on the detector collects the electrons and holes on opposite sites of the strip, where the charge is collected on aluminum contacts and a signal is measured. Energy is not deposited in these contacts, and therefore they constitute to a so called dead layer. \\
\\
The detectors are square $5\times 5$ \SI{}{cm} and with their $16\times 16$ strips, they have an effective gird of  256 pixels of \SI{9}{mm}. 
4 of the 6 detectors have a thickness of \SI{60}{\mu m} and a dead layer of \SI{100}{nm} dead layer. These detectectors are the ones called Det1, Det3, Det4 and DetU in the setup seen on figure \ref{fig:opstilling}. The other 2 detectors (Det2 and DetD) where \textcolor{red}{INSERT THICKNESS HERE}

\section{AUSAlib and ROOT}
ROOT is an object oriented C++ framework that is designed primarily for data analysis in high-energy and nuclear physics. It was created at CERN in 1995, and has since grown and become the dominant analysis software at both CERN and many other nuclear and particle physics laboratories. 
ROOT was designed to handle large amounts of data with high computing efficiency. \\
ROOT makes an intelligent data structure by creating a "Tree" with the class \texttt{TTree}. This tree will then have "branches" which corresponds to some variable of the given detection event, such as the energy of the front strip or identity of the detector. This TTree then allows for reading of an individual branch, while ROOT takes care of the memory management. One can also store a TTree to the disk in the form a .root file. \\

ASUALib is a tool that build on top of ROOT. It was created for the subatomic group at Aarhus University.

