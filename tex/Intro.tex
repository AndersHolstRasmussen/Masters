\chapter{Introduction}
In this thesis i will make a thorough analysis of the the delayed \be-decay of \isotope[8]Li. The first part of the thesis will will be governing getting data from a detector sorted into an appropriate data structure that can be analyzed. 
The second part is a more physics analysis of the \al-particle pair that is produced, and their correlation to the \be-particle.
Lastly a discussion of the result and setup will be conducted. 

\begin{enumerate}
	\item We are interrested in the decay, and want to describe what is happening.
	\item Different levels which gives a continuum of 2+ state.
	\item \isotope[8]{Li} har spin +2
	\item We expect 2+ structure in \isotope[8]{Be}
	\item There exists earlier measurements of \isotope[8]B	made in the AU group.
	\item \isotope[8]B is the mirror core?? And has been measured very precisely.
	\item Measure Li with same precision.
	\item A beautiful spectrum from the alphas.
	\item Look at a single alpha.
	\item Look at coincidence.
	\item Better to look at both and add, recoil will give widening.
	\item read bataratjaa
	\item We are interrested in beta alfa angles for \isotope[12]B
	\item People have measured this angle precisely, because they where looking for deviations in the standard model.
\end{enumerate}

We want to examine the \be-delayed decay of \isotope[8]Li, which is an unstable isotope of Lithium. Lithium normallyu occurs stable as \isotope[6]Li and \isotope[7]Li, with the latter being the more abundant with 92.5\% of all atoms. The longest living radioactive lithium isotope is \isotope[8]Li, with a half-life of \SI{839}{ms} \note{ref}. 

\li is an interesting atom, as it will decay into \isotope[8]Be, which is a constituent in the triple-alpha process in stellar astrophysics. \isotope[8]Be creates a bottleneck for the creation of heavier elements, as it is very short lived, and have a half-life of \SI{1e-16}{s}.

\li will also decay into \ber at different energy levels, which will give a continuum of the 2+ state from \li. 

The Sub-atomic group at Aarhus University has previously measured decay of the mirror nuclei, \isotope[8]B. Since there exists very precise measurement, we want to optain similar precise measurements of \li. 